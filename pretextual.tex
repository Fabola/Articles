% ----------------------------------------------------------
% Neste arquivo edita-se a DEDICATÓRIA, os AGRADECIMENTOS, a EPÍGRAFE, os RESUMOS, a LISTA DE ABREVIATURAS E SIGLAS e a LISTA DE SÍMBOLOS
% ----------------------------------------------------------

% ---
% Dedicatória
% ---
\begin{dedicatoria}
   \vspace*{\fill}
	\begin{flushright}
		Este trabalho é dedicado às crianças adultas que \\
		quando pequenas, sonharam em se tornar cientistas.
	\end{flushright}
\end{dedicatoria}
% ---

% ---
% Agradecimentos
% ---
\begin{agradecimentos}
Os agradecimentos principais são direcionados à Gerald Weber, Miguel Frasson,
Leslie H. Watter, Bruno Parente Lima, Flávio de Vasconcellos Corrêa, Otavio Real
Salvador, Renato Machnievscz\footnote{Os nomes dos integrantes do primeiro
projeto abn\TeX\ foram extraídos de
\url{http://codigolivre.org.br/projects/abntex/}} e todos aqueles que
contribuíram para que a produção de trabalhos acadêmicos conforme
as normas ABNT com \LaTeX\ fosse possível.

Agradecimentos especiais são direcionados ao Centro de Pesquisa em Arquitetura
da Informação\footnote{\url{http://www.cpai.unb.br/}} da Universidade de
Brasília (CPAI), ao grupo de usuários
\emph{latex-br}\footnote{\url{http://groups.google.com/group/latex-br}} e aos
novos voluntários do grupo
\emph{\abnTeX}\footnote{\url{http://groups.google.com/group/abntex2} e
\url{http://www.abntex.net.br/}}~que contribuíram e que ainda
contribuirão para a evolução do \abnTeX.

\end{agradecimentos}
% ---

% ---
% Epígrafe
% ---
\begin{epigrafe}
    \vspace*{\fill}
	\begin{flushright}
		A gente quer passar um rio a nado, e passa; mas vai dar na \\
		outra banda é num ponto muito mais abaixo, bem diverso \\
		do em que primeiro se pensou. Viver nem não é muito \\
		perigoso? \\
		\textit{Grande Sertão: Veredas -- Guimarães Rosa}
	\end{flushright}
\end{epigrafe}
% ---

% ---
% RESUMOS
% ---

% resumo em português
\setlength{\absparsep}{18pt} % ajusta o espaçamento dos parágrafos do resumo
\begin{resumo}

 \SingleSpacing{\citarautor~{\bfseries \rmfamily \fontsize{12}{12} \imprimirtitulo}. \imprimirdata. \thelastpage p. \tipotrabalhocurto~(\cursorealizado) -- \ppg, \imprimirinstituicao, \imprimirlocal.}

 Segundo a \citeonline[3.1-3.2]{NBR6028:2003}, o resumo deve ressaltar o
 objetivo, o método, os resultados e as conclusões do documento. A ordem e a extensão
 destes itens dependem do tipo de resumo (informativo ou indicativo) e do
 tratamento que cada item recebe no documento original. O resumo deve ser
 precedido da referência do documento, com exceção do resumo inserido no
 próprio documento. (\ldots) As palavras-chave devem figurar logo abaixo do
 resumo, antecedidas da expressão Palavras-chave:, separadas entre si por
 ponto e finalizadas também por ponto.

 \textbf{Palavras-chave}: \textit{latex. abntex. editoração de texto}.
\end{resumo}

% resumo em inglês
\begin{resumo}[Abstract]

 \SingleSpacing{\citarautor~{\bfseries \rmfamily \fontsize{12}{12} \imprimirtitulo}. \imprimirdata. \thelastpage p. \tipotrabalhocurto~(\cursorealizado) -- \ppg, \imprimirinstituicao, \imprimirlocal.}

 \begin{otherlanguage*}{english}
   This is the english abstract.

   \vspace{\onelineskip}

   \noindent
   \textbf{Keywords}: \textit{latex. abntex. text editoration}.
 \end{otherlanguage*}
\end{resumo}

% resumo em francês
%\begin{resumo}[Résumé]

%  \SingleSpacing{\citarautor~{\bfseries \rmfamily \fontsize{12}{12} \imprimirtitulo}. \imprimirdata. \thelastpage p. \tipotrabalhocurto~(\cursorealizado) -- \ppg, \imprimirinstituicao, \imprimirlocal.}


% \begin{otherlanguage*}{french}
%    Il s'agit d'un résumé en français.
%
%   \textbf{Mots-clés}: latex. abntex. publication de textes.
% \end{otherlanguage*}
%\end{resumo}

% resumo em espanhol
%\begin{resumo}[Resumen]

%  \SingleSpacing{\citarautor~{\bfseries \rmfamily \fontsize{12}{12} \imprimirtitulo}. \imprimirdata. \thelastpage p. \tipotrabalhocurto~(\cursorealizado) -- \ppg, \imprimirinstituicao, \imprimirlocal.}


% \begin{otherlanguage*}{spanish}
%   Este es el resumen en español.
%
%   \textbf{Palabras clave}: latex. abntex. publicación de textos.
% \end{otherlanguage*}
%\end{resumo}
% ---

% ---
% inserir lista de ilustrações
% ---
\renewcommand{\listfigurename}{Lista de Figuras} % #modificado: Renomeia de Lista de Ilustrações para Lista de Figuras. Colocando em ppgec.cls não funcionou.
% ---
\pdfbookmark[0]{\listfigurename}{lof}
\listoffigures*
\cleardoublepage
% ---

% ---
% inserir lista de quadros
% ---
%\pdfbookmark[0]{\listofquadrosname}{loq}
%\listofquadros*
%\cleardoublepage
% ---

% ---
% inserir lista de tabelas
% ---
\pdfbookmark[0]{\listtablename}{lot}
\listoftables*
\cleardoublepage
% ---

% ---
% LISTA DE ABREVIATURAS E SIGLAS
% ---
\begin{siglas}
  \item[ABNT] Associação Brasileira de Normas Técnicas
  \item[abnTeX] ABsurdas Normas para TeX
\end{siglas}
% ---

% ---
% LISTA DE SÍMBOLOS
% ---
\begin{simbolos}
  \item[$ \Gamma $] Letra grega Gama
  \item[$ \Lambda $] Lambda
  \item[$ \zeta $] Letra grega minúscula zeta
  \item[$ \in $] Pertence
\end{simbolos}
% ---

% ---
% inserir o sumario
% ---
\pdfbookmark[0]{\contentsname}{toc}
\tableofcontents*
\cleardoublepage
% ---