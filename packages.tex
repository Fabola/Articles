% ---
% Pacotes básicos
% ---
\usepackage{times}		  	    % Usa a fonte Times apenas para texto - usar:
\usepackage{mathptmx}         % para times tbm nas equações
\usepackage[T1,LGRx,T1]{fontenc}  		% Selecao de codigos de fonte.
\usepackage[utf8]{inputenc}		% Codificacao do documento (conversão automática dos acentos)
\usepackage{color}			    	% Controle das cores
\usepackage{graphicx}		    	% Inclusão de gráficos
\graphicspath{{./figuras/}}   % Caminho para a pasta que contém os arquivos das Figuras
\usepackage{microtype} 		  	% para melhorias de justificação
\usepackage{fancyhdr}
\usepackage{amssymb,amsmath,amsfonts,textcomp,bm}
% ---

% ---
% Pacotes adicionais, usados apenas no âmbito do Modelo Canônico do abnteX2
% ---
\usepackage{lipsum}				% para geração de dummy text
% ---

% ---
% Pacotes de citações
% ---
\usepackage[brazilian,hyperpageref]{backref}	 % Paginas com as citações na bibl
\usepackage[alf]{abntex2cite}	% Citações padrão ABNT


% ---
% Incluído por Augusto B. Borges
% ---
\usepackage{caption}
% Tabelas
\usepackage{array}								% Elementos extras para formatação de tabelas
\usepackage{booktabs}							% Tabelas com qualidade de publicação
\usepackage{longtable}							% Para criar tabelas maiores que uma página
\usepackage{lscape}								% adicionar tabelas e figuras como landscape
% Notas de rodapé
\usepackage{footnote}							% Lidar com notas de rodapé em diversas situações
\makesavenoteenv{tabular}						% Notas criadas nas tabelas ficam no fim das tabelas

% ---
% CONFIGURAÇÕES DE PACOTES
% ---

% ---
% Configurações do pacote backref
% Usado sem a opção hyperpageref de backref
\renewcommand{\backrefpagesname}{Citado na(s) página(s):~}
% Texto padrão antes do número das páginas
\renewcommand{\backref}{}
% Define os textos da citação
\renewcommand*{\backrefalt}[4]{
	\ifcase #1 %
		Nenhuma citação no texto.%
	\or
		Citado na página #2.%
	\else
		Citado #1 vezes nas páginas #2.%
	\fi}%
% ---