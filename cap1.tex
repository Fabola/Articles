% ----------------------------------------------------------
\chapter{Introdução}\label{introducao}
% ----------------------------------------------------------

Este documento e seu código-fonte são exemplos de referência de uso da classe
\textsf{abntex2} e do pacote \textsf{abntex2cite}. O documento 
exemplifica a elaboração de tese ou dissertação no formato requerido pelo PPGEC-UFRGS, no qual seu formato não corresponde exatamente ao produzido conforme a ABNT NBR 14724:2011 \emph{Informação e documentação
- Trabalhos acadêmicos - Apresentação}.

Este modelo é baseado no \abnTeX\ mas modificado para o nosso Programa de Pós-Graduação e, futuramente, será colocado nos requisitos do \abnTeX. Uma lista completa das normas
observadas pelo \abnTeX\ original é apresentada em \citeonline{abntex2classe}.

Sinta-se convidado a participar do projeto original \abnTeX! Acesse o site do projeto em
\url{http://www.abntex.net.br/}. Também fique livre para conhecer,
estudar, alterar e redistribuir o trabalho do \abnTeX, desde que os arquivos
modificados tenham seus nomes alterados e que os créditos sejam dados aos
autores originais, nos termos da ``The \LaTeX\ Project Public
License''\footnote{\url{http://www.latex-project.org/lppl.txt}}. Isso permite que futuras versões do \abnTeX~não se tornem automaticamente
incompatíveis com as customizações promovidas. Consulte
\citeonline{abntex2-wiki-como-customizar} para mais informações.

Este documento deve ser utilizado como complemento dos manuais do \abnTeX\ 
\cite{abntex2classe,abntex2cite,abntex2cite-alf} e da classe \textsf{memoir}
\cite{memoir}. 

Espero, sinceramente, que o este modelo aprimore a qualidade do trabalho que
você produzirá, de modo que o principal esforço seja concentrado no principal:
na contribuição científica.


Augusto Bopsin Borges\footnote{modificada da mensagem original da Equipe \abnTeX~e de Lauro César Araujo}
  

