% ----------------------------------------------------------
% Informações do Aluno e outros dados 
% ----------------------------------------------------------

\titulo{Título Completo da Dissertação ou Tese}
\autor{Fulano de Tal}
\newcommand{\citarautor}{Tal, F. D.}
\newcommand{\emailautor}{fulano.de.tal@ufrgs.br}
\local{Porto Alegre}
\data{201X}
\newcommand{\datacompleta}{[dia] de [mês] de 201X}
\orientador{Nome do Orientador}
\newcommand{\titorientador}{Dr. pela Universidade de Origem}
%\coorientador{Equipe \abnTeX}
\instituicao{Universidade Federal do Rio Grande do Sul}
\newcommand{\abrevinstituicao}{UFRGS}
\newcommand{\escola}{Escola de Engenharia}
\newcommand{\ppg}{Programa de Pós-Graduação em Engenharia Civil}
\newcommand{\abrevppg}{PPGEC}
\tipotrabalho{Tese de Doutorado}
\newcommand{\tipotrabalhocurto}{Tese}
\newcommand{\cursorealizado}{Doutorado em Engenharia}
\newcommand{\tituloobtido}{Doutor em Engenharia}
\newcommand{\areaconcentracao}{Geotecnia}
% O preambulo deve conter o tipo do trabalho, o objetivo,
% o nome da instituição e a área de concentração
\preambulo{\imprimirtipotrabalho~apresentada ao \ppg~da \imprimirinstituicao~como parte dos requisitos para obtenção do título de \tituloobtido.}

% Dados do coordenador do curso
\newcommand{\coordenador}{Nilo Consoli}
\newcommand{\titcoordenador}{Ph.D. pela Concordia University, Canadá}

% Membros da Banca e titulação
\newcommand{\tratbancaum}{Prof.}
\newcommand{\bancaum}{Luís Carlos Prestes}
\newcommand{\origembancaum}{UFRGS}
\newcommand{\titbancaum}{Ph.D. pela Universidade de Origem, País}

\newcommand{\tratbancadois}{Prof.}
\newcommand{\bancadois}{Leonel de Moura Brizola}
\newcommand{\origembancadois}{UFRGS}
\newcommand{\titbancadois}{Dr. pela Universidade Federal do Rio Grande do Sul}

\newcommand{\tratbancatres}{Prof.}
\newcommand{\bancatres}{Getúlio Vargas}
\newcommand{\origembancatres}{UFRGS}
\newcommand{\titbancatres}{Dr. pela Universidade Federal do Rio Grande do Sul}

\newcommand{\tratbancaquatro}{Profa.}
\newcommand{\bancaquatro}{Anita Garibaldi}
\newcommand{\origembancaquatro}{UFSC}
\newcommand{\titbancaquatro}{Dra. pela Universidade Federal de Santa Catarina}
% ---


% NÃO MEXER DAQUI PARA BAIXO

% ---
% Configurações de aparência do PDF final

% alterando o aspecto da cor azul
\definecolor{blue}{RGB}{41,5,195}

% informações do PDF
\makeatletter
\hypersetup{
     	%pagebackref=true,
		pdftitle={\@title},
		pdfauthor={\@author},
    	pdfsubject={\imprimirpreambulo},
	    pdfcreator={LaTeX with abnTeX2},
		pdfkeywords={abnt}{latex}{abntex}{abntex2}{trabalho acadêmico},
		colorlinks=true,       		% false: boxed links; true: colored links
    	linkcolor=black,       	% color of internal links
    	citecolor=black,     		% color of links to bibliography
    	filecolor=black,     		% color of file links
		urlcolor=black,
		bookmarksdepth=4
}
\makeatother
% ---

% ---
% Posiciona figuras e tabelas no topo da página quando adicionadas sozinhas
% em um página em branco. Ver https://github.com/abntex/abntex2/issues/170
\makeatletter
\setlength{\@fptop}{5pt} % Set distance from top of page to first float
\makeatother
% ---

% ---
% Possibilita criação de Quadros e Lista de quadros.
% Ver https://github.com/abntex/abntex2/issues/176
%
\newcommand{\quadroname}{Quadro}
\newcommand{\listofquadrosname}{Lista de quadros}

\newfloat[chapter]{quadro}{loq}{\quadroname}
\newlistof{listofquadros}{loq}{\listofquadrosname}
\newlistentry{quadro}{loq}{0}

% configurações para atender às regras da ABNT
\setfloatadjustment{quadro}{\centering}
\counterwithout{quadro}{chapter}
\renewcommand{\cftquadroname}{\quadroname\space}
\renewcommand*{\cftquadroaftersnum}{\hfill--\hfill}

\setfloatlocations{quadro}{hbtp} % Ver https://github.com/abntex/abntex2/issues/176
% ---

% ---
% Espaçamentos entre linhas e parágrafos
% ---
\OnehalfSpacing % Definição do PPGEC-UFRGS
% O tamanho do parágrafo é dado por:
\setlength{\parindent}{0cm} % Definição do PPGEC-UFRGS

% Controle do espaçamento entre um parágrafo e outro:
\setlength{\parskip}{0.42336cm} % este é o espaçamento no modelo PPGEC, equivalente a 12pt (1pt = 0,03528cm)
% tente também \onelineskip

% ---
% compila o indice
% ---
\makeindex
% ---

% ----
% Início do documento
% ----
